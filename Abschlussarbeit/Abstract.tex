UML state machine \cite{harel_naamad_1996} ,also known as UML statechart, is used to describe and the design of complex discrete-events systems \cite{harel_1987,harel_naamad_1996}.
Understanding state machine diagrams is essential for object-oriented software design.
So teaching this knowledge is an important part of many computer science modules.
For the purpose of E-learning,  we want to design a tool in Haskell for the automatic generation of UML state machine tasks.
The paper thoroughly discusses testing properties for our existing state diagram drawing tool in Haskell \cite{jun_hao_tan} and how to generate random test cases satisfying these criteria.
Firstly, we discuss in detail the essential constraints and invariants for a state diagram and how this model-checking tool is realized in Haskell. 
Then we introduce the random diagram generator. 
Finally, randomly generated expressions are fed to the program to discover errors with our software and visualize the diagram.

\keywords{ UML state diagram, QuickCheck, E-Learning, Haskell, Fuzzing}
